\documentclass{article}
\usepackage[T1]{fontenc}
\usepackage[utf8]{inputenc}
\usepackage[english,shorthands=off]{babel}
\usepackage[autostyle]{csquotes}
\MakeOuterQuote{"}
\usepackage{lmodern}
\usepackage[default]{lato}
\usepackage[
tracking=true,
expansion=true,
protrusion=true,
babel
]{microtype}
\usepackage{tikz}

\usetikzlibrary{%
  arrows,
  arrows.meta,
  calc,
  decorations,
  decorations.shapes,
  decorations.text,
  decorations.pathreplacing,
  fit,
  positioning,
  shapes.geometric,
}

\definecolor{tableaublack}{HTML}{181818}
\definecolor{tableaublue}{HTML}{1F77B4}
\definecolor{tableauorange}{HTML}{FF7F0E}
\definecolor{tableaugreen}{HTML}{2CA02C}
\definecolor{tableaured}{HTML}{D62728}
\definecolor{tableaupurple}{HTML}{9467BD}
\definecolor{tableaubrown}{HTML}{8C564B}
\definecolor{tableaumagenta}{HTML}{E377C2}
\definecolor{tableaugray}{HTML}{7F7F7F}
\definecolor{tableaugrey}{HTML}{7F7F7F}
\definecolor{tableaulime}{HTML}{BCBD22}
\definecolor{tableaucyan}{HTML}{17BECF}
\definecolor{tableaulightblue}{HTML}{AEC7E8}
\definecolor{tableaulightorange}{HTML}{FFBB78}
\definecolor{tableaulightgreen}{HTML}{98DF8A}
\definecolor{tableaulightred}{HTML}{FF9896}
\definecolor{tableaulightpurple}{HTML}{C5B0D5}
\definecolor{tableaulightbrown}{HTML}{C49C94}
\definecolor{tableaulightmagenta}{HTML}{F7B6D2}
\definecolor{tableaulightgray}{HTML}{C7C7C7}
\definecolor{tableaulightgrey}{HTML}{C7C7C7}
\definecolor{tableaulightlime}{HTML}{DBDB8D}
\definecolor{tableaulightcyan}{HTML}{9EDAE5}

\usepackage[graphics,active,tightpage]{preview}
\PreviewEnvironment{tikzpicture}
\setlength{\PreviewBorder}{1mm}
\newlength{\roundedcorners}
\setlength{\roundedcorners}{5pt}
\newlength{\distancex}
\setlength{\distancex}{10pt}
\newlength{\distancey}
\setlength{\distancey}{15pt}
\newlength{\componentwidth}

\tikzset{
  box/.style = {
    align=center,
    inner sep=3pt,
    minimum height=20pt,
    shape=rectangle,
    text=black,
  },
  connect/.style = {
    arrows={Stealth[inset=0pt, angle=60:7pt, line width=0]-Stealth[inset=0pt, angle=60:7pt, line width=0]},
    line width=3pt,
    draw=tableaublue!50!black,
  },
  interface/.style={
    box,
    text=white,
    fill=black,
    opacity=0.5,
  },
  component/.style={
    box,
    append after command={
      \pgfextra
      \clip[rounded corners=\roundedcorners] (\tikzlastnode.south west) rectangle (\tikzlastnode.north east);
      \endpgfextra
    }
  },
  outline/.style={
    draw=black,
    inner sep=0,
    rounded corners=\roundedcorners,
  },
}

\begin{document}
\begin{tikzpicture}
  \setlength{\componentwidth}{450pt}
  \begin{scope}
    \node[
    component,
    fill=tableaulightblue,
    minimum width=\componentwidth,
    minimum height=100pt,
    align=center
    ] (core) {
      \begin{minipage}{0.9\componentwidth}
        \centering
        \vspace{20pt}
        \texttt{openscm.core},
        \texttt{openscm.parameters},
        \texttt{openscm.parameter\_views},\\
        \texttt{openscm.regions},
        \texttt{openscm.units}
      \end{minipage}
      \\[0.5em]
      \begin{minipage}{0.75\componentwidth}
        \centering
        Provides the actual functionality to run a particular simple
        climate model as well as getting and setting its parameter
        values. Mapping of parameter names as well as conversion of
        time frames and units is done internally.

        The main entry point is the \texttt{openscm.OpenSCM} class which
        represents a model run.
      \end{minipage}
    };

    \node[
    interface,
    anchor=north,
    minimum width=\componentwidth,
    ] at (core.north) {OpenSCM core interface};
  \end{scope}
  \node[outline,fit=(core)] {};

  \setlength{\componentwidth}{130pt}
  \begin{scope}
    \node[
    component,
    fill=tableaulightpurple,
    minimum width=\componentwidth,
    minimum height=70pt,
    below right=\distancey and 0pt of core.south west,
    ] (magiccadapter) {};

    \node[
    interface,
    anchor=north,
    minimum width=\componentwidth,
    ] at (magiccadapter.north) {\texttt{openscm.adapter.Adapter}};

    \node[
    anchor=south,
    minimum width=\componentwidth-10pt,
    yshift=5pt,
    minimum height=40pt,
    rounded corners=\roundedcorners,
    draw=black,
    fill=tableaupurple,
    text=white
    ] at (magiccadapter.south) {
      \begin{minipage}{40pt}
        \centering
        MAGICC\\model
      \end{minipage}
    };
  \end{scope}
  \node[outline,fit=(magiccadapter)] {};

  \setlength{\componentwidth}{130pt}
  \begin{scope}
    \node[
    component,
    fill=tableaulightpurple,
    minimum width=\componentwidth,
    minimum height=70pt,
    right=\distancex of magiccadapter,
    ] (hectoradapter) {};

    \node[
    interface,
    anchor=north,
    minimum width=\componentwidth,
    ] at (hectoradapter.north) {\texttt{openscm.adapter.Adapter}};

    \node[
    anchor=south,
    minimum width=\componentwidth-10pt,
    yshift=5pt,
    minimum height=40pt,
    rounded corners=\roundedcorners,
    draw=black,
    fill=tableaupurple,
    text=white
    ] at (hectoradapter.south) {
      \begin{minipage}{40pt}
        \centering
        Hector\\model
      \end{minipage}
    };
  \end{scope}
  \node[outline,fit=(hectoradapter)] {};

  \setlength{\componentwidth}{130pt}
  \begin{scope}
    \node[
    component,
    fill=tableaulightpurple,
    minimum width=\componentwidth,
    minimum height=70pt,
    right=\distancex of hectoradapter,
    ] (fairadapter) {};

    \node[
    interface,
    anchor=north,
    minimum width=\componentwidth,
    ] at (fairadapter.north) {\texttt{openscm.adapter.Adapter}};

    \node[
    anchor=south,
    minimum width=\componentwidth-10pt,
    yshift=5pt,
    minimum height=40pt,
    rounded corners=\roundedcorners,
    draw=black,
    fill=tableaupurple,
    text=white
    ] at (fairadapter.south) {
      \begin{minipage}{40pt}
        \centering
        FaIR\\model
      \end{minipage}
    };
  \end{scope}
  \node[outline,fit=(fairadapter)] {};

  \node[box,right=\distancex of fairadapter] (modeldots) {\Huge ...};

  \draw[connect] let \p1=(core.south), \p2=(magiccadapter.north) in (\x2,\y1) -- (\x2,\y2);
  \draw[connect] let \p1=(core.south), \p2=(hectoradapter.north) in (\x2,\y1) -- (\x2,\y2);
  \draw[connect] let \p1=(core.south), \p2=(fairadapter.north) in (\x2,\y1) -- (\x2,\y2);

  \setlength{\componentwidth}{130pt}
  \begin{scope}
    \node[
    component,
    fill=tableaulightred,
    minimum width=\componentwidth,
    minimum height=100pt,
    above right=\distancey and 89.75pt of core.north west,
    ] (dataframe) {};

    \node[
    interface,
    anchor=north,
    minimum width=\componentwidth,
    ] at (dataframe.north) {\texttt{openscm.scmdataframe}};

    \node[
    interface,
    fill=none,
    text=black,
    opacity=1,
    yshift=10pt,
    anchor=south,
    minimum width=\componentwidth,
    ] at (dataframe.south) {
      \begin{minipage}{0.9\componentwidth}
        \centering
        Provides a tabular view to input and output parameters
        including methods to read and write a standardized file
        format.
      \end{minipage}
    };
  \end{scope}
  \node[outline,fit=(dataframe)] {};
  \draw[connect] let \p1=(core.north), \p2=(dataframe.south) in (\x2,\y1) -- (\x2,\y2);

  \setlength{\componentwidth}{130pt}
  \begin{scope}
    \node[
    component,
    fill=tableaulightred,
    minimum width=\componentwidth,
    minimum height=100pt,
    right=\distancex of dataframe,
    ] (ensemble) {};

    \node[
    interface,
    anchor=north,
    minimum width=\componentwidth,
    ] at (ensemble.north) {\texttt{openscm.ensemble}};

    \node[
    interface,
    fill=none,
    text=black,
    opacity=1,
    yshift=17pt,
    anchor=south,
    minimum width=\componentwidth,
    ] at (ensemble.south) {
      \begin{minipage}{0.9\componentwidth}
        \centering
        Provides functions for ensemble runs e.g. perturbed parameter
        experiments.
      \end{minipage}
    };
  \end{scope}
  \node[outline,fit=(ensemble)] {};
  \draw[connect] let \p1=(core.north), \p2=(ensemble.south) in (\x2,\y1) -- (\x2,\y2);

  \setlength{\componentwidth}{450pt}
  \begin{scope}
    \node[
    component,
    fill=tableaulightgreen,
    minimum width=\componentwidth,
    minimum height=50pt,
    above right=\distancey and -89.75pt of dataframe.north west,
    ] (users) {};

    \node[
    anchor=west,
    xshift=5pt,
    minimum width=115pt,
    minimum height=40pt,
    rounded corners=\roundedcorners,
    draw=black,
    fill=tableaugreen,
    text=white,
    ] (tool) at (users.west) {
      \begin{minipage}{210pt}
        \centering
        OpenSCM Tool\\{\small Command line interface}
      \end{minipage}
    };

    \node[
    minimum width=115pt,
    minimum height=40pt,
    rounded corners=\roundedcorners,
    draw=black,
    fill=tableaugreen,
    text=white,
    right=\distancex of tool,
    xshift=-4pt,
    ] (endusers) {
      \begin{minipage}{210pt}
        \centering
        Other libraries\\{\small IAMs, ...}
      \end{minipage}
    };
  \end{scope}
  \node[outline,fit=(users)] {};

  \draw[connect] let \p1=(users.south), \p2=(core.north), \p3=($(core.north west)+(50pt,0)$) in (\x3,\y1) -- (\x3,\y2);
  \draw[connect] let \p1=(users.south), \p2=(core.north), \p3=($(core.north east)+(-50pt,0)$) in (\x3,\y1) -- (\x3,\y2);
  \draw[connect] let \p1=(users.south), \p2=(dataframe.north) in (\x2,\y1) -- (\x2,\y2);
  \draw[connect] let \p1=(users.south), \p2=(ensemble.north) in (\x2,\y1) -- (\x2,\y2);
\end{tikzpicture}
\end{document}
